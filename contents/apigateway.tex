\chapter{API Gateway}
This chapter discusses the specifications of the API gateway.

\section{Definitions}

\begin{teitemize}
    \item \textbf{Service}\\
    A service refers to the definition of the API of a microservice. An example is ``\texttt{email-service}''. This shows that a service exists with that name.
    \item \textbf{Service identifier}\\
    Multiple instances of one service can exist. Each instance is identified by its own \teurl[GUID]{https://en.wikipedia.org/wiki/Universally\_unique\_identifier}. A generated GUID is practically unique.
    \item \textbf{Endpoint}\\
    The endpoint
\end{teitemize}
    
    \section{Routing}
Requests are made to \texttt{http://api-hostname/api/}\teunderline{\texttt{service-name}}\texttt{/}\teunderline{\texttt{endpoint}}.

The API gateway re-routes this request to the approperiate server. If no server is running, a HTTP status code of 503 (Unavailable) is returned.

Query parameters (such as \texttt{/endpoint?value=true}) and POST bodies are forwarded to the service as well.

\section{Service discovery}
\subsection{Service requirements}
Certain requirements are put on services which register themselves at the API gateway.

\begin{teenumerate}
    \item \textbf{Registering}\\
    A client registers themselves by sending a request POST to \texttt{\textbackslash{}register}. The POST request must contain the following body:
    \begin{lstlisting}[style=teunknown]
{
    "ServiceName": "[The name of the service]",
    "Guid": "[The generated GUID]",
    "BaseUrl": "[The base url this service is reachable at]"
}
    \end{lstlisting}
    It is OK---and even encouraged---to send duplicate register requests. This is useful when the API gateway has restarted (and lost its services) but the service is still running.
    \item \textbf{Ping}\\
    The service must have a \texttt{/status} endpoint which returns the status of the service. This status must be:
    \begin{teitemize}
        \item \texttt{200} when the service is running OK.
        \item \texttt{504} when the service is (partly) not available due to issues.
    \end{teitemize} 
    If this endpoint is not reachable, the service is assumed to be irreparably crashed and is removed from the service discovery. The service must register itself again.
    \item \textbf{Unregistering}\\
    Unregistering is \teunderline{optional} but preferred. If the service stops without unregistering it will eventually be removed because \texttt{/status} is unreachable. A service can unregister itself by sending a POST request to \texttt{/unregister} with the following body:
    \begin{lstlisting}[style=teunknown]
{
    "Guid": "[The GUID of this service]"
}
    \end{lstlisting} 
\end{teenumerate}